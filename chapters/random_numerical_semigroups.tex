% All contribution chapters should follow a similar structure, with a
% mini-introduction and overview at the beginning and a conclusion at the
% end bookmarking a structured presentation of the contribution. This can be
% largely based on your publications.

\chapter{Random Numerical Semigroups}\label{chap:randnumsems}

We present three distinct models of random numerical semigroups. The first model employs a uniform distribution for its generators and is the most extensively studied. The second model is inspired in the Erdös-Rényi approach for random graphs. The third model is a new approach we introduce, characterized by a fixed Frobenius number.

\section{Box Model}\label{sec:randomsmpgs:intro}

Let $n, T \in \NN$. We consider the set of points in the set
\[G(T) = \{a \in \NN : \mathrm{gcd}(a) = 1, \; |a|_\infty \leq T\}.\]
This is the set of 



\section{ER-type model}

In order to use similar methods that can be applied to the Erdös-Rényi model for random graphs, a similar model for random numerical semigroups was proposed in \cite{de2018random}. 

\begin{definition}
    For $p \in [0, 1]$ and $M \in \NN$, an ER-type random numerical semigroup $S(M, p)$ is a probability space over the set of semigroups $S = \langle\mathcal{A}\rangle$ with $\mathcal{A} \subseteq \{1,...,M\}$, determined by
    \[\Pr[n \in \mathcal{A}] = p,\]
    with these events mutually independent.
\end{definition}

In other words, a semigroup $S(M, p)$ is obtained by using the following procedure:
\begin{enumerate}
    \item Initialize a set $\mathcal{A} = \{0\}$.
    \item From step $1$ to $M$, add $i$ to $\mathcal{A}$ with probability $p$, independently of the other steps.
    \item Return the semigroup $S = \langle\mathcal{A}\rangle$.
\end{enumerate}

The main result of \cite{de2018random} is the following theorem.  

\begin{theorem}\label{thm:ermodel}
    
\end{theorem}

They also provide the following bounds.
\begin{itemize}
    \item Test. 
\end{itemize}

We prove the threshold function for co-finiteness. 

\begin{theorem}
    
\end{theorem}

When $p$ is a constant and let $M$ tend to infinity, we think of the ER-type model as 

\section{Downward model}\label{sec:contrib1:theme2}

In relation with Wilf's conjecture \ref{conj:smgps:wilf}, we propose a new model for random numerical semigroups that fixes the Frobenius number.

The main tool for proving the other parts of Theorem \ref{thm:ermodel} is a correspondence between irreducible numerical semigropus with a simplicial complex. 

\begin{definition}
    
\end{definition}

